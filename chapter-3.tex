\chapter{Indeksowanie danych}

Aby zrozumieć, w jaki sposób dane przechowywane są indeksie Lucene, wyjaśnimy na początku kilka pojęć:
\begin{itemize}
 \item \emph{Term} -- jest podstawową jednostka wyszukiwania i indeksowania. Term jest ściśle związany z polem, w jakim występuje. Dlatego pojedynczy term utożsamiamy z parą nazwa pola--słowo.
 \item \emph{Pole} -- jest to zbiór termów, oznaczony identyfikującą go nazwą i posiadający określony typ. Pole posiada też pewne parametry wskazujące, jak jego zawartość (lista należących do niego termów) powinna być traktowana podczas indeksowania i wyszukiwania. Określa np. czy zawartość powinna być tokenizowania lub przechowywana w indeksie. Wartości tych parametrów są na ogół zależne od wybranego typu pola.
 \item \emph{Dokument} -- jest zbiorem pól. Jest identyfikowany przez swój numer, unikalny w obrębie segmentu. Dokumenty w Lucene nie mają narzuconej, sztywnej struktury. Oznacza to, że w obrębie jednego indeksu mogą występować dokumenty o całkowicie różnych polach. 
\end{itemize}

Dokumenty dodawane do Lucene są przechowywane w indeksie odwróconym, który mówiąc w uproszczeniu, jest strukturą mapującą termy na listy numerów dokumentów zawierających dany term. Na pojedynczy indeks może składać się wiele \emph{segmentów}. Segment jest sam w sobie w pełni funkcjonalnym indeksem -- podział indeksu na mniejsze jednostki pozwala m.in. na zrównoleglenie procesu wyszukiwania.

\section{Typy pól}

Lucene pozwala na utworzenie pól następujących typów:
\begin{enumerate}
 \item pola tekstowe: \texttt{TextField} i \texttt{StringField}. Pola obydwóch typów są indeksowane, różnica między nimi polega na tym, że \texttt{StringField} nie jest dzielone na tokeny -- cała jego zawartość traktowana jest jako pojedynczy token. Wobec tego nadaje się np. do przechowywania numerów identyfikacyjnych, nazwisk czy nazw państw. \texttt{TextField} jest polem indeksowanym i analizowanym -- zazwyczaj występuje jako główne pole dokumentu, przechowujące większą część treści. Opcjonalnie, obydwa typy pól tekstowych mogą być przechowywane w indeksie (\emph{stored fields}).
 \item pola numeryczne: \texttt{IntField}, \texttt{LongField}, \texttt{DoubleField}, \texttt{FloatField}. Pozwalają na indeksowanie wartości liczbowych oraz na późniejsze wyszukiwanie i filtrowanie po zakresach wartości. Każde pole typu numerycznego jest indeksowane jako drzewo trie.
 \item pole ``przechowywane'', \texttt{StoredField}: pole, którego pełna zawartość jest przechowywana w indeksie i może być odzyskana podczas wyszukiwania. Tego typu pole zazwyczaj stosuje się do zapisu takich informacji jak tytuł czy abstrakt dokumentu, jego zawartość nie jest jednak ograniczona do tekstu -- \texttt{StoredField} może także przechowywać liczby lub wartości binarne.
 \item pola typu DocValues: \texttt{NumericDocValuesField}, \texttt{BinaryDocValuesField}, \texttt{SortedDocValuesField}, \texttt{SortedSetDocValuesField} -- jest to nowy typ pola, wprowadzony w wersji 4.0 Lucene. Zawartości pól typu DocValues nie są włączane do indeksu odwróconego -- a zatem nie można po nich wyszukiwać. Są natomiast przydatne podczas sortowania lub grupowania wyników wyszukiwania. W przeciwieństwie do indeksu odwróconego, który w uproszczeniu jest strukturą mapującą termy na listy indeksów dokumentów, DocValues zaimplementowane są jako struktura mapująca indeksy dokumentów na zadane wartości (typu zależnego od wybranej implementacji pola). DocValues są jedynie optymalizującym dodatkiem do Lucene -- funkcje, do jakich są wykorzystywane mogłyby zostać spełnione przez pola innych typów. Zaimplementowane są jednak w sposób pozwajający na szybkie pobieranie przechowywanych wartości dla każdego dokumentu.
\end{enumerate}
