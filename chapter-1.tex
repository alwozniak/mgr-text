\chapter{Wprowadzenie}

W ciągu ostatnich kilkudziesięciu lat obserwujemy zjawisko zwane ``eksplozją danych'': ilość informacji zgromadzonych przez ludzkość rośnie eksponencjalnie. Dostępność informacji nie byłaby jednak do niczego przydatna -- bez możliwości efektywnego ich przeszukiwania. Dodatkowo, większość danych, które codziennie produkujemy nie posiada wyraźnej struktury.  Wyszukiwanie interesujących fragmentów w takich zbiorach nie jest więc problemem trywialnym. Dlategoteż tak ważne stało się tworzenie narzędzi to ułatwiających -- wyszukiwarek. 

Celem niniejszej pracy jest przedstawienie biblioteki Apache Lucene pozwalającej tworzyć wyszukiwarki dokumentów tekstowych. W kolejnych rozdziałach opisane zostaną: funkcjonalności oferowane przez Lucene, jej architektura oraz przytoczone zostaną przykłady jej wykorzystania.

\section{Apache Lucene}

Lucene jest jedną z najpopularniejszych bibliotek wyszukiwania tekstowego. Jest napisana w całości w języku Java i udostępniona jest na licencji open source.

Apache Lucene skupia w sobie cztery główne podprojekty: 
\begin{itemize}
 \item \emph{Lucene Core}, będący faktyczną implementacją silnika wyszukiwania. Na opisie tego właśnie podprojektu skupimy się w tej pracy. W dalszej części tekstu termin ``Lucene'' będzie odnosił się do Lucene Core.
 \item \emph{Solr} jest implementacją serwera wyszukiwania, wykorzystującą Lucene Core jako silnik. Solr, poza wyszukiwaniem tekstowym, udostępnia takie funkcjonalności jak integracja z bazą danych, operowanie na popularnych formatach dokumentów (Word, PDF) czy wyszukiwanie po współrzędnych geograficznych.
 \item \emph{Open Relevance Project} dostarcza narzędzi do testowania trafności wyników wyszukiwania.
 \item \emph{PyLucene} jest rozszerzeniem Pythona pozwalającym na korzystanie z możliwości Lucene.
\end{itemize}

Dokumentacja Lucene Core wymienia m.in. następujące cechy charakterystyczne biblioteki:
\begin{itemize}
 \item szybkie, skalowalne indeksowanie dokumentów -- z możliwościami indeksowania inkrementalnego wykazującego się podobnymi właściwościami,
 \item rozmiar indeksu nie przekraczający 20-30\% rozmiaru pierwotnych dokumentów,
 \item wyszukiwanie połączone z sortowaniem wyników zgodnie z malejącą trafnością dopasowania do zapytania,
 \item dostępność wielu rodzajów zapytań,
 \item zdefiniowane przez użytkownika sortowanie wyników,
 \item możliwość konfiguracji formatu plików indeksu (architektura oparta o kodeki).
\end{itemize}

Lucene Core, pomimo, że napisana w Javie, może być wykorzystana także w innych językach. Umożliwiają to liczne projekty towarzyszące, m.in. CLucene (implementacja w C++), Lucene.NET (portowanie biblioteki na platformę .NET) i LuceneKit (implementacja w języku Objective-C wraz ze wsparciem dla natywnej dla środowiska MacOS X biblioteki Cocoa i jej open source'owej implementacji GNUstep). Wszystkie implementacje Lucene gwarantują kompatybilność utworzonych przez nie indeksów.

\section{Przykład wykorzystania Lucene}

Poniżej zamieszczony jest prosty program wykorzystujący Lucene. Indeksuje on dokumenty z zadanego katalogu i pozwala na wyszukiwanie tych z nich, które zawierają podane przez użytkownika słowo kluczowe. Bazę dokumentów stanowi zbiór pięciu anglojęzycznych książek, pobranych ze strony Projektu Gutenberg \cite{gutenberg}.

Oto przykładowy wynik uruchomienia programu:
\begin{lstlisting}[escapeinside={@}{@}]
@\label{result-1:enter}@Enter a single word query:
thing
@\label{result-1:query}@Documents matching the query: contents:thing
@\label{result-1:begin}@Lewis Carroll - Alice's Adventures in Wonderland.txt
James M. Barrie - Peter Pan.txt
Emily Bronte - Wuthering Heights.txt
Arthur Conan Doyle - The Adventures of Sherlock Holmes.txt
@\label{result-1:end}@Jane Austen - Pride and Prejudice.txt
\end{lstlisting}

Po zaindeksowaniu dokumentów program pyta użytkownika o zapytanie, po którym ma odbywać się wyszukiwanie (linia \ref{result-1:enter}). Gdy użytkownik wprowadzi słowo kluczowe, następuje proces wyszukiwania. Program wypisuje zapytanie (linia \ref{result-1:query}) a następnie -- tytuły pasujących dokumentów (linie \ref{result-1:begin}-\ref{result-1:end}). Zauważmy, że zapytanie składa się z dwóch elementów: pola, po ktorym ma odbywać się wyszukiwanie oraz słowa kluczowego, oddzielonych dwukropkiem.

Zapytanie użyte w powyższym przykładzie jest jednym z najpopularniejszych słów języka angielskiego -- nie dziwi więc fakt, że w wyniku otrzymaliśmy wszystkie dokumenty. Jeśli użyjemy mniej popularnego słowa, lista pasujących dokumentów będzie krótsza.

\begin{lstlisting}[escapeinside={@}{@}]
Enter a single word query:
suspect
Documents matching the query: contents:suspect
Jane Austen - Pride and Prejudice.txt
Emily Bronte - Wuthering Heights.txt
Arthur Conan Doyle - The Adventures of Sherlock Holmes.txt
\end{lstlisting}

\subsection{Struktura programu}

Zamieszczona poniżej metoda \texttt{main()} ilustruje przebieg programu.

\begin{lstlisting}[escapeinside={@}{@}]
/**
 * Maximum number of results returned by IndexSearcher instance.
 */
@\label{main:MAX_DOCS}@private static final int MAX_DOCS = 10;

/**
 * Relative path to a folder containing documents for indexing.
 */
private static final String CORPUS_FOLDER_PATH = "../books";

/**
 * An index directory.
 */
@\label{main:directory}@private static final Directory directory = new RAMDirectory();

public static void main(String[] args) throws IOException {
  @\label{main:createIndex}@createIndex(directory, CORPUS_FOLDER_PATH);
  @\label{main:obtainQuery}@Query query = obtainQuery();
  @\label{main:createIndexSearcher}@IndexSearcher searcher = createIndexSearcher(directory);
  // Perform search.
  @\label{main:search}@TopDocs topDocs = searcher.search(query, MAX_DOCS);
  @\label{main:printResults}@printResults(query, searcher, topDocs);
}
\end{lstlisting}

W linii \ref{main:createIndex} tworzony jest indeks. Metoda \texttt{createIndex()} jako parametry przyjmuje obiekt typu \texttt{Directory} oraz ścieżkę do folderu zawierającego dokumenty do zaindeksowania. \texttt{Directory} jest interfejsem reprezentującym katalog zawierający pliki indeksu. W opisywanym przykładzie posługujemy się indeksem trzymanym w pamięci podręcznej, stąd korzystamy z implementacji \texttt{RAMDirectory} (linia \ref{main:directory}).

Następnie, w linii \ref{main:obtainQuery} tworzymy obiekt typu \texttt{Query}, reprezentujący zapytanie. Klasą odpowiedzialną za przeszukiwanie indeksu jest \texttt{IndexSearcher} -- tworzony w linii \ref{main:createIndexSearcher}. Faktyczne wyszukiwanie odbywa się poprzez wykonanie polecenia \texttt{searcher.search(query, MAX\_DOCS)} w linii \ref{main:search} (\texttt{MAX\_DOCS} -- linia \ref{main:MAX_DOCS} -- jest maksymalną liczbą wyników, które może zwrócić proces wyszukiwania). Program kończy się wypisaniem wyników w linii \ref{main:printResults}.

\subsection{Tworzenie indeksu}

Metoda \texttt{createIndex()} ilustruje proces tworzenia indeksu: na początku tworzony jest obiekt typu \texttt{IndexWriter} (linia \ref{createIndex:indexWriter}) oraz lista dokumentów (linia \ref{createIndex:documents}). Po dodaniu dokumentów do zapisu (linia \ref{createIndex:add}) następuje commit oraz zamknięcie obiektu \texttt{indexWriter} (linie \ref{createIndex:commit} -- \ref{createIndex:close}).

\begin{lstlisting}[label=createIndex,escapeinside={@}{@}]
private static void createIndex(Directory directory, String path) throws IOException {
  @\label{createIndex:indexWriter}@IndexWriter indexWriter = createIndexWriter(directory);
  @\label{createIndex:documents}@List<Document> documents = createDocuments(path);
  @\label{createIndex:add}@indexWriter.addDocuments(documents);
  @\label{createIndex:commit}@indexWriter.commit();
  @\label{createIndex:close}@indexWriter.close();
}

private static IndexWriter createIndexWriter(Directory directory) throws IOException {
  @\label{indexWriter:config}@IndexWriterConfig config = new IndexWriterConfig(Version.LUCENE_44, new StandardAnalyzer(Version.LUCENE_44));
  return new IndexWriter(directory, config);
}
\end{lstlisting}

\texttt{IndexWriter} jest klasą odpowiedzialną za zapis struktur indeksu do wskazanego katalogu. Poza obiektem typu \texttt{Directory}, \texttt{IndexWriter} przyjmuje w konstruktorze obiekt konfiguracji: \texttt{IndexWriterConfig} (linia \ref{indexWriter:config}), który z kolei potrzebuje instancji typu \texttt{Analyzer} (wykorzystaną tutaj implementacją jest \texttt{StandardAnalyzer}). \texttt{Analyzer} definiuje sposób ekstrakcji termów z dokumentu.

Dokumenty tworzone są na podstawie plików zawartych we wskazanym folderze.

\begin{lstlisting}
private static List<Document> createDocuments(String path) throws IOException {
  List<Document> documents = new ArrayList<>();
  File documentsFolder = new File(path);
  if (!documentsFolder.isDirectory()) {
    throw new IOException("Specified path does not point to a directory.");
  }
  File[] files = documentsFolder.listFiles();
  for (File file : files) {
    if (!file.isDirectory() && !file.isHidden() && file.exists() && file.canRead()) {
      Document document = createDocument(file);
      documents.add(document);
    }
  }
  return documents;
}
\end{lstlisting}

Pojedynczy dokument (\texttt{Document}) jest zbiorem pól. Każde pole ma określony typ (wyspecyfikowany przez implementację abstrakcyjnej klasy \texttt{Field}), nazwę (podawaną jako pierwszy parametr konstruktora) oraz zawartość. Opcjonalnie zawartość ta może być zapisana w plikach indeksu -- aby można było później ją zwrócić w wyniku wyszukiwania. Przykład takiego pola znajduje się w linii \ref{createDocument:store} poniższego listingu: podczas tworzenia pola \texttt{filename} jako trzeci parametr konstruktora podane jest \texttt{Field.Store.YES} .

\begin{lstlisting}[label=createDocument,escapeinside={@}{@}]
/**
 * Field names.
 */
public static final String CONTENT_FIELD = "contents";
public static final String FILE_NAME_FIELD = "filename";

private static Document createDocument(File file) throws FileNotFoundException {
  Document document = new Document();
  document.add(new TextField(CONTENT_FIELD, new FileReader(file)));
  @\label{createDocument:store}@document.add(new StringField(FILE_NAME_FIELD, file.getName(), Field.Store.YES));
  return document;
}
\end{lstlisting}

\subsection{Tworzenie zapytania}

Zapytanie jest tworzone na podstawie wczytanego z linii poleceń tekstu. Korzystamy z klasy \texttt{TermQuery} implementującej abstrakcyjne \texttt{Query} -- reprezentującej zapytanie składające się z pojedynczego termu. Ponieważ w Lucene termy identyfikowane są przez parę \emph{pole:term}, posługujemy się także nazwą pola, w którym odbywać się będzie wyszukiwanie (linia \ref{obtainQuery:termQuery}).

\begin{lstlisting}[label=obtainQuery,escapeinside={@}{@}]
private static Query obtainQuery() {
  System.out.println("Enter a single word query:");
  Scanner scanner = new Scanner(System.in);
  String queryString = scanner.nextLine().trim();
  scanner.close();
  @\label{obtainQuery:termQuery}@return new TermQuery(new Term(CONTENT_FIELD, queryString));
}
\end{lstlisting}

\subsection{Wyszukiwanie}

Obiektem odpowiedzialnym za wykonywanie zapytań na indeksie jest \texttt{IndexSearcher}. Sposób tworzenia go ilustruje poniższy listing.

\begin{lstlisting}
private static IndexSearcher createIndexSearcher(Directory directory) {
  IndexReader reader = null;
  try {
    reader = DirectoryReader.open(directory);
  } catch (IOException e) {
    e.printStackTrace();
  }
  return new IndexSearcher(reader);
}
\end{lstlisting}

\subsection{Wypisywanie wyników}

Metoda \texttt{search()} klasy \texttt{IndexSearcher} zwraca obiekt typu \texttt{TopDocs}, który m.in. zawiera listę najlepiej pasujących do zapytania dokumentów. Listing poniżej pokazuje sposób uzyskania i wykorzystania tej listy (dla każdego dokumentu obecnego w wyniku, program wypisuje na konsolę nazwę pliku odpowiadającemu temu dokumentowi).

\begin{lstlisting}
private static void printResults(Query query, IndexSearcher searcher, TopDocs topDocs) throws IOException {
  System.out.println("Documents matching the query: " + query);
  for (ScoreDoc scoreDoc : topDocs.scoreDocs) {
    Document document = searcher.doc(scoreDoc.doc);
    String documentTitle = document.get(FILE_NAME_FIELD);
    System.out.println(documentTitle);
  }
}
\end{lstlisting}
